% !TeX root = RJwrapper.tex
\title{Análisis multivariante empleando correspondencias canónicas -
COVID19 - Ecuador}
\author{by William F. Tandazo-Vargas}

\maketitle

\abstract{%
An abstract of less than 150 words.
}

\hypertarget{introducciuxf3n}{%
\subsection{Introducción}\label{introducciuxf3n}}

El 11 de marzo del 2020, La organización mundial de la salud (OMS)
declara pandemia el brote del virus SARS-CoV-2, causante del COVID-19.
Esta pandemia desato una serie de crisis dentro de diferentes aspectos
sociales, económicos y políticos de las naciones del mundo. Los sistemas
sanitarios tanto de potencias como de países con economías emergentes se
mostraron frágiles e incapaces de satisfacer la demanda de pacientes
infectados con esta enfermedad. Se ha mostrado que para el caso de
algunos países que conforman el bloque G7 las acciones tomadas durante
el brote causaron un impacto significativo en el desarrollo del virus
SARS-CoV-2 \citep{ZHANG2020109829}.

En Ecuador, el caso cero fue reportado el 29 de febrero del 2020. Se
determino que la paciente posiblemente lo contrajo desde España, país el
cual la paciente llego dos semanas antes de confirmar su estado. Es
desde este punto que se tiene constancia de la evolución del virus en
Ecuador, situación que se complico a las pocas semanas después, lo que
provocó el confinamiento total del país un 17 de marzo del 2020
\citep{Haro2020}. Las cifras de fallecidos después de esos días se
dispararon llegando a sus picos más grandes los días del 22 de marzo al
9 de junio del 2020. Se pudo demostrar luego que los reportes generales
descritos por el ministerio de salud pública del Ecuador (MSP) no
contemplaban o ajustaban al verdadero impacto que genero esta pandemia
desde los meses de febrero a octubre \citep{CEVALLOSVALDIVIEZO2021297}.

El análisis de datos se volvió fundamental en esta pandemia, se
convirtió en un arma más para combatir el virus. Casos como el de Taiwán
ponen en evidencia que el manejo e integración de los metadatos y su
análisis previnieron la saturación de los hospitales, ofreciendo un
servicio y manejo apropiado de los ciudadanos durante los días más
fuertes del brote \citep{Chen2020} . Con el análisis de los datos se
logró determinar el impacto de las comorbilidades en pacientes que
tienen el virus, lo que permitió darles una prioridad respecto a los
demás que no poseen enfermedades como Diabetes o problemas
cardiovasculares \citep{Li2020} \citep{Pal2020} \citep{Zhang2020}, al
igual demostrar que este virus afecta indiferentemente del sexo, pero se
vuelve más crítico en pacientes de grupos etarios mayores a 65 años.\\
Para el caso de Ecuador, el subregistro de los factores o variables que
tuvieron relación con el impacto de la pandemia no han permitido aclarar
el desarrollo de esta. El estudio multivariante de las distintas
variables de salud como camas disponibles, ventilación, admisiones
dentro de los hospitales y las unidades de cuidado intensivo (UCI) tiene
un poder predictivo importante al momento de medir la mortalidad del
virus \citep{Fatima2020}. Al igual que en materia económica, esto sirvió
para evidenciar el ``hoyo'' que genero la pandemia en la economía global
de los mercados \citep{Sharma2020}. Pese a que los datos de las
distintas variables inherentes asociadas al COVID-19 para Ecuador no han
sido debidamente liberadas es posible realizar un estudio multivariante
con las existentes, como en el caso de Nepal \citep{Devkota2021}.

Gracias a las medidas de restricción de movilización humana dentro de
Ecuador se ha mostrado que ciertos agentes contaminantes como el NO2
disminuyeron hasta en 5.6 veces menos con respecto a la media comparando
el año de la pandemia con el 2018 y el 2019 \citep{Zambrano2020}. El
estudio de estos factores es importante puesto que estos agentes en
altas concentraciones tienen una incidencia directa en la salud
respiratoria de los ciudadanos. También, estudios realizados en Singapur
han demostrado que este tipo de variables meteorológicas como la calidad
del aire, temperatura y nivel de precipitación diaria fueron
significativos en días donde hubieron varios casos confirmados de
COVID-19 \citep{LORENZO2021111024}.

Este estudio tiene como objetivo realizar un estudio multivariante
tomando variables de registros publicos y de subregistros; variables
como de mortalidad, condiciones de vida, alertas ciudadanas, casos
confirmados de COVID-19 y variables de carácter meteorológicas dentro de
las fechas del {[}por confirmar{]}. Para esto se usará un modelo de
análisis de correspondencia canónica para encontrar la naturaleza de las
relaciones existentes entre estas variables obtenidas en los comienzos
de la pandemia de COVID-19.

\hypertarget{muxe9todos-y-materiales.}{%
\subsection{Métodos y Materiales.}\label{muxe9todos-y-materiales.}}

\hypertarget{reduciuxf3n-de-dimensiones.}{%
\subsubsection{Redución de
dimensiones.}\label{reduciuxf3n-de-dimensiones.}}

Con el inicio de la pandemia surgen distintas formas de analizar el
impacto en el que esta afecta a un determinado sector. Estas variables
son numerosas para distintos ámbitos en los que se quiera enfocar, es
decir, si nuestro estudio posee p-variables explicativas entonces estas
serán de dimensión p.~Reducir las dimensiones será fundamental para
nuestro análisis, llevar a una expresión condensada de estas que
explique su interrelación nos ayudará a concluir los objetivos de este
estudio. Tomar que variables son las que más aportan dentro de un
estudio ha sido tema de investigación durante mucho tiempo dentro del
campo estadístico. Existen dos distintos enfoques para esto, uno que
queda en manos del investigador el cual de todas las covariables
disponibles toma un grupo de estas según su criterio, estas siendo las
que el cree son las menos redundantes y las más significativas al
momento de explicar el fenómeno que estudia. El segundo enfoque, el cual
es el que desarrollaremos más de fondo en este estudio, será el de
reducción de dimensiones suficiente. Este enfoque producirá una
combinación lineal del grupo de covariables tratadas. De lo anterior, se
pueden producir varias combinaciones lineares de un grupo de
covariables, la meta es decir que combinación o combinaciones son las
que mejor explican el efecto que se estudia. Para el análisis de
correspondencias, este paso es crucial y permite hacerse con un costo de
perdida de información mínima, nuestro objetivo será el restringir esta
mínima perdida de información para que este explique el máximo de
información.

\hypertarget{ordenamiento-en-espacio-reducido.}{%
\subsubsection{Ordenamiento en espacio
reducido.}\label{ordenamiento-en-espacio-reducido.}}

Al reducir la dimensión de los datos es necesario que estos estén
ordenados alrededor de una serie de ejes ortogonales reducida, estos de
manera descendiente siendo el primero el que más información provee de
esta estructura. Esto se hace por medio de la extracción de valores
propios asociado a la matriz de datos de nuestras variables. Si esta
matriz de datos es de nxp, n observaciones en p variables estudiadas,
estas n observaciones pueden ser representadas por grupos en un espacio
p dimensional. Este grupo de datos usualmente se acumula más en ciertas
direcciones y es plano en otras dentro de nuestro espacio p dimensional
y no necesariamente este tomara el curso de alguna de las variables.
Donde más esta alargado esta nube de puntos es donde reside la mayor
varianza de este grupo, será el mayor gradiente que representan nuestros
datos. Este será nuestro primer eje que se extraerá y es el que más
información dota nuestro análisis, el siguiente eje será el segundo más
impórtate solo si este es ortogonal al primero. El numeró de ejes que
interpretan la estructura de los datos se determina por el método a
tomar que en nuestro caso ser el de análisis de correspondencia.

\hypertarget{anuxe1lisis-de-correspondencia.}{%
\subsubsection{Análisis de
Correspondencia.}\label{anuxe1lisis-de-correspondencia.}}

El análisis de correspondencia (AC) es una técnica estadística la cual
es útil cuando en la matriz de datos se tienen filas y columnas de
naturaleza categórica. En particular, se usa cuando se tiene tablas de
contingencia la cual contiene frecuencias numéricas de los perfiles,
produciendo un análisis gráfico más simple y formal, haciendo que su
interpretación sea eficiente. Esta técnica buscara la mejor
representación de las variables de la matriz en un plano de baja
dimensión, los cuales llamaremos ejes que de manera ordinal se
organizaran siendo el primero el que mejor explique la asociación entre
los perfiles filas y columnas. Los otros factores trataran explicar la
mayor parte del residuo que no explico su inmediato anterior,
organizándose de manera descendiente. Este método buscara encontrar
estas dimensiones tal que la variabilidad geométrica de la nube de
puntos (también llamada inercia) sea máxima.\\
Esencialmente, el análisis de correspondencia tiene 3 fases: La
transformación inicial de la matriz de datos, la descomposición en
valores singulares de la transformación y el cambio de escala de los
vectores proprios resultantes. Del primer paso, la transformación de la
matriz se lleva acabo de la siguiente forma:

\[ H=S^{-\frac{1}{2}}XC^{-\frac{1}{2}} \ \ \ \ \ \ (1)\]

La matriz \(H\) tiene los valores transformados de la matriz \(X\). Las
matrices \(S^{-\frac{1}{2}}\) y \(C^{-\frac{1}{2}}\) son matrices
diagonales que contienen la raíz cuadrada de las marginales totales
reciprocas tanto de las dimensiones de la fila y columna. Esta
transformación remueve los efectos de las magnitudes que se formas de
las diferencias entre marginales totales. En el caso de tablas de
contingencia lo anterior seria equivalente a remover los valores
esperados del estadístico Chi-cuadrado del en el modelo de
independencias.

La segunda fase constaría en obtener los vectores característicos de
nuestra matriz, mediante descomposición de valores singulares.

\[ X_{n,m}= U_{n,m}d_{m,m}V^{T}_{n,m} \ \ \ \ \ \ (2) \] Donde:

\[ U_{n,m} : \ es \ la \ matriz \ de \ la \ dimensión \ fila. \]

\[ V^{T}_{n,m} : \ es \ la \  matriz \ de \ la \ dimensión \ columna. \]

\[ d_{m,m} : \ es \ la \ matriz \ diagonal \ de \ valores \ propios.\]

Las matrices \(U\), \(V\) y \(d\) se pueden encontrar directamente con
una descomposición de valores singulares en la matriz de datos \(X\) o
indirectamente mediante la extracción de los valores propios de las
matrices de productos cruzados de \(X\). Como última fase, las columnas
tanto de las matrices \(U\) y \(V\) deben ser reescaladas para obtener
los valores óptimos, también llamados valores canónicos. El resultado de
esto da un ordenamiento donde se preservará la distancia chi-cuadrado
(\(\chi ^{2}\)) entre los perfiles. Gráficamente el resultado de un
análisis de correspondencias puede ser descrito por medio de un
``Biplot'' el cual grafica puntos del perfil fila \(x_i\) y puntos del
perfil columna \(y_j\) tal que los productos escalares entre los
vectores de fila y columna se aproximan a los elementos correspondientes
de los datos de la matriz lo más cerca posible en un espacio reducido.

\hypertarget{anuxe1lisis-de-correspondencia-muxfaltiple.}{%
\subsubsection{Análisis de Correspondencia
Múltiple.}\label{anuxe1lisis-de-correspondencia-muxfaltiple.}}

El análisis de Correspondencia (ACM) visto en el punto anterior es el
caso más sencillo cuando se tiene dos variables categóricas. El análisis
de correspondencia múltiple involucra 3 o más variables dentro del
conjunto de variables a estudiar. El análisis de correspondencia
múltiple equivale a realizar un análisis de correspondencia simple, pero
a la matriz indicadora de nuestros datos. En estas matrices indicadoras
cada fila representa un caso y las columnas representan todas las
categorías de las variables. Para el caso de variables de naturaleza
continua estas pueden ser redefinidas por categorías, con una variable
por categoría. Empleando un análisis por medio del producto cruz o
matriz de Burt de las variables indicadoras se puede llegar a la
solución óptima.

\hypertarget{anuxe1lisis-de-correspondencia-canuxf3nica.}{%
\subsubsection{Análisis de Correspondencia
Canónica.}\label{anuxe1lisis-de-correspondencia-canuxf3nica.}}

Como hemos visto, el análisis de correspondencia nos ayuda a visualizar
los datos de una matriz dentro de un espacio de dimensión reducida,
donde la inercia de esta nube de puntos será la más optima. Un análisis
de correspondencia canónico (ACC) se puede emplear cuando encontramos
estas dimensiones a partir del AC, pero con la condición de que estas
son una combinación lineal de variables explicativas adicionales.
Podemos ver esto como un problema de regresión lineal múltiple, solo que
en vez de hacer regresión sobre las variables explicativas lo haremos a
las dimensiones de estas variables explicativas. El ACC restringe el
espacio donde este buscara los ejes principales óptimos de un espacio
total, el complemento de este será un espacio no restringido el cual
podría también tener la solución óptima. El porcentaje total de inercia
estará repartido entre estos dos espacios, siendo el restringido el
lugar donde se encontraría la inercia máxima. En consecuencia, a lo
anterior el ACC explicara un poco menos del total de inercia existente.
Para interpretar el modelo de manera gráfica se sigue la de un AC
clásico y se sigue la misma explicación que la del ``Biplot''.

\hypertarget{datos}{%
\subsection{Datos}\label{datos}}

\bibliography{RJreferences.bib}

\address{%
William F. Tandazo-Vargas\\
ESPOL\\%
line 1\\ line 2\\
%
%
%
\\\href{mailto:wtandazo@espol.edu.ec}{\nolinkurl{wtandazo@espol.edu.ec}}
}
